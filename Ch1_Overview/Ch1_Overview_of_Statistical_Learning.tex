% Options for packages loaded elsewhere
\PassOptionsToPackage{unicode}{hyperref}
\PassOptionsToPackage{hyphens}{url}
%
\documentclass[
]{article}
\usepackage{amsmath,amssymb}
\usepackage{lmodern}
\usepackage{ifxetex,ifluatex}
\ifnum 0\ifxetex 1\fi\ifluatex 1\fi=0 % if pdftex
  \usepackage[T1]{fontenc}
  \usepackage[utf8]{inputenc}
  \usepackage{textcomp} % provide euro and other symbols
\else % if luatex or xetex
  \usepackage{unicode-math}
  \defaultfontfeatures{Scale=MatchLowercase}
  \defaultfontfeatures[\rmfamily]{Ligatures=TeX,Scale=1}
\fi
% Use upquote if available, for straight quotes in verbatim environments
\IfFileExists{upquote.sty}{\usepackage{upquote}}{}
\IfFileExists{microtype.sty}{% use microtype if available
  \usepackage[]{microtype}
  \UseMicrotypeSet[protrusion]{basicmath} % disable protrusion for tt fonts
}{}
\makeatletter
\@ifundefined{KOMAClassName}{% if non-KOMA class
  \IfFileExists{parskip.sty}{%
    \usepackage{parskip}
  }{% else
    \setlength{\parindent}{0pt}
    \setlength{\parskip}{6pt plus 2pt minus 1pt}}
}{% if KOMA class
  \KOMAoptions{parskip=half}}
\makeatother
\usepackage{xcolor}
\IfFileExists{xurl.sty}{\usepackage{xurl}}{} % add URL line breaks if available
\IfFileExists{bookmark.sty}{\usepackage{bookmark}}{\usepackage{hyperref}}
\hypersetup{
  pdftitle={Overview of Statistical Learning},
  pdfauthor={Seung Hyun Sung},
  hidelinks,
  pdfcreator={LaTeX via pandoc}}
\urlstyle{same} % disable monospaced font for URLs
\usepackage[margin=1in]{geometry}
\usepackage{graphicx}
\makeatletter
\def\maxwidth{\ifdim\Gin@nat@width>\linewidth\linewidth\else\Gin@nat@width\fi}
\def\maxheight{\ifdim\Gin@nat@height>\textheight\textheight\else\Gin@nat@height\fi}
\makeatother
% Scale images if necessary, so that they will not overflow the page
% margins by default, and it is still possible to overwrite the defaults
% using explicit options in \includegraphics[width, height, ...]{}
\setkeys{Gin}{width=\maxwidth,height=\maxheight,keepaspectratio}
% Set default figure placement to htbp
\makeatletter
\def\fps@figure{htbp}
\makeatother
\setlength{\emergencystretch}{3em} % prevent overfull lines
\providecommand{\tightlist}{%
  \setlength{\itemsep}{0pt}\setlength{\parskip}{0pt}}
\setcounter{secnumdepth}{-\maxdimen} % remove section numbering
\ifluatex
  \usepackage{selnolig}  % disable illegal ligatures
\fi

\title{Overview of Statistical Learning}
\author{Seung Hyun Sung}
\date{11/4/2021}

\begin{document}
\maketitle

\hypertarget{contents}{%
\subsubsection{Contents}\label{contents}}

\begin{itemize}
\tightlist
\item
  1A Introduction to Regression Models
\item
  1B Conditional Expectation (Harvard Open Source Lecture)
\item
  1C Dimensionality and Structured Models
\item
  1D Model Selection and Bias-Variance Trade-off
\item
  1E Least Squares and Nearest Neighbours
\item
  1F K-Nearest Neighbours in R
\end{itemize}

\begin{quote}
In essence, statistical learning refers to a set of approaches for
estimating \(\ f\). In this chapter we outline some of the key
theortical concepts that arise in estimating \(\ f\), as well as tools
for evaluating the estimates obtained. --- pg 17 \emph{An Introduction
to Statistical Learning with Apllications in R}.
\end{quote}

\hypertarget{a-introduction-to-regression-models}{%
\subsection{1A Introduction to Regression
Models}\label{a-introduction-to-regression-models}}

Depending on the family of Regression model and its complexity of
\(\ f(x)\), we may be able to understand how each compnent \(\ X_j\)
affects \(\ Y\), in what particular fashion. Depending on the task
(target variable \(\ Y\)) will vary in weight of importance between the
interpretation and accuracy. Hence, the phase we are with predicting or
defining the task it is important to take this in account before
designing and selecting a model.

The notation for ideal Regression function:

\[ f(x) = E(Y|X = x) \] Immediately it emphasis that this regression
function gives conditional expectation of \(\ Y|X\). Is there an ideal
\(\ f(X)\)? The ideal regression function means the expected value
(average) of \(\ Y\) given \(\ X\). In section 1B we will explore more
on this conditional expectation, its useful properties, and geometric
interpretation.

For example, if \(\ X\) had three components \(\ x \in R^3\) It is going
to be a conditional expectation of \(\ Y\) given three particular
instances of these three components of \(\ X\).

\[
f(x) = f(x_1, x_2, x_3) = E(Y|X_1 = x_1, X_2 = x_2, X_3 = x_3) 
\] The question given to the function is that at particular point X with
three coordinates, \(\ X_1, X_2, X_3\), what is good value for the
function at that point of instances.

Meaning of the ideal or optimal regression function:

\begin{itemize}
\item
  Conditional Average - \(\ E(Y)\) would be the averages of \(\ Y's\) at
  these coordinates, and the regression function would do that at all
  points in the plane.
\item
  Ideal means with regard to a loss function, the particular choice of
  the function \(\ f(x)\) will minimise the sum of squared errors. I.e.
\end{itemize}

\[
f(x) = E(Y|X = x)\ is\ the\ function\ that\ minimises\ E((Y - g(X))^2|X = x)\ over\ all\ functions\ g\ at\ all\ points\ X = x 
\] * At each point \(\ X\), there will be mistakes.

\$\$

E((Y - \hat{f}(X))\^{}2\textbar X = x) = (f(x) - \hat{f}(x))\^{}2 +
Var(\epsilon)

\$\$

\end{document}
